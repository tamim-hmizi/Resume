\documentclass[10pt, letterpaper]{article}

% ================= PACKAGES =================
\usepackage[
    ignoreheadfoot,
    top=2cm,
    bottom=2cm,
    left=2cm,
    right=2cm
]{geometry}

\usepackage{titlesec}
\usepackage{enumitem}
\usepackage{hyperref}
\usepackage{needspace}
\usepackage{iftex}
\usepackage{charter}
\usepackage{microtype} % FIXES encoding & copy artifacts

% ================= ATS / ENCODING =================
\ifPDFTeX
  \input{glyphtounicode}
  \pdfgentounicode=1
  \usepackage[T1]{fontenc}
  \usepackage[utf8]{inputenc}
  \usepackage{lmodern}
\fi

\pagestyle{empty}
\setcounter{secnumdepth}{0}
\setlength{\parindent}{0pt}
\renewcommand\labelitemi{$\bullet$}

\titleformat{\section}{\bfseries\large}{}{0pt}{}[\titlerule]
\titlespacing{\section}{0pt}{6pt}{6pt}

\newenvironment{highlights}{
\begin{itemize}[leftmargin=12pt, itemsep=2pt]
}{
\end{itemize}
}

\hypersetup{
  colorlinks=true,
  urlcolor=black
}

% ================= DOCUMENT =================
\begin{document}

% ================= HEADER =================
\begin{center}
{\fontsize{22}{22}\selectfont \textbf{Tamim Hmizi}} \\[4pt]
{\large Consultant Infrastructure \,|\, Cloud \& DevOps \,|\, SaaS \& Full-Stack} \\[6pt]
Ariana, Tunisie \;|\;
\href{mailto:tamimhmizi@icloud.com}{tamimhmizi@icloud.com} \;|\;
+216 21 611 816 \;|\;
\href{https://linkedin.com/in/tamimhmizi}{linkedin.com/in/tamimhmizi} \;|\;
\href{https://github.com/tamim-hmizi}{github.com/tamim-hmizi} \;|\;
\href{https://tamim-hmizi.github.io/Portfolio/}{Portfolio}
\end{center}

% ================= PROFILE =================
\section{Profil Professionnel}
\textbf{Consultant Infrastructure / Ingénieur Cloud \& DevOps} disposant d’une expérience concrète au sein d’une entreprise spécialisée en \textbf{Cloud, DevOps et développement d’applications SaaS et Full-Stack}. Solide expertise dans la conception, le déploiement, l’automatisation, la supervision et l’exploitation d’\textbf{infrastructures hybrides et cloud-native} supportant des produits logiciels modernes. Compétences confirmées en CI/CD, observabilité, réseaux, systèmes de sauvegarde et livraison applicative de bout en bout en environnements professionnels et de production.

% ================= SKILLS =================
\section{Compétences Techniques}

\textbf{Cloud \& Plateformes :} Azure, AWS, GCP, OpenStack, DigitalOcean, Cloud hybride, IaaS, PaaS \\
\textbf{Conteneurs \& Orchestration :} Docker, Docker Compose, Kubernetes, Helm \\
\textbf{DevOps \& Automatisation :} Jenkins, GitHub Actions, GitLab CI, Terraform, Ansible, Pipelines CI/CD, Infrastructure as Code (IaC), Nexus \\
\textbf{Supervision \& Observabilité :} Grafana, Prometheus, ELK Stack, CloudWatch, Alerting, Métriques, Tableaux de bord \\
\textbf{Sécurité \& Qualité :} Pare-feu Fortinet, VPN, Segmentation réseau, IAM, SonarQube, Trivy, OWASP \\
\textbf{Sauvegarde \& PRA :} Commvault (sauvegardes, tests de restauration, analyse d’échecs, reporting) \\
\textbf{Systèmes \& Identité :} Linux, Windows Server, Active Directory, Microsoft 365 \\
\textbf{Langages de Programmation :} Python, Bash, Java, JavaScript, TypeScript, C, C++, C\#, PHP, Assembleur, SQL \\
\textbf{Frameworks Backend :} FastAPI, Django, Flask, Express.js, Spring Boot, Node.js, APIs REST \\
\textbf{Frameworks Frontend :} Angular, React, Next.js, Vue.js, HTML, CSS \\
\textbf{Bases de Données :} MongoDB, MySQL, PostgreSQL, DynamoDB, Redis \\
\textbf{Gestion de Code :} Git, GitHub, GitLab, Bitbucket, Gitea

% ================= EXPERIENCE =================
\section{Expérience Professionnelle}

\textbf{Consultant Infrastructure — RFC} \hfill Octobre 2025 -- Présent \\
Ariana, Tunisie
\begin{highlights}
\item Exploitation et support d’environnements cloud et hybrides pour la fourniture de services SaaS et d’infrastructures d’entreprise.
\item Supervision des ressources de calcul, de stockage et d’infrastructure via des tableaux de bord Grafana.
\item Mise en place et maintenance de pipelines de métriques basés sur Prometheus pour la détection proactive des incidents.
\item Déploiement et administration de Gitea sur des machines virtuelles Linux en tant que plateforme Git auto-hébergée.
\item Configuration, maintenance et dépannage de pare-feu Fortinet (politiques de sécurité, segmentation, routage, NAT).
\item Exploitation de l’infrastructure de sauvegarde Commvault : exécution, analyse des échecs, validation des restaurations et reporting.
\item Animation de formations techniques et d’ateliers clients autour de l’infrastructure et des stratégies de sauvegarde.
\item Gestion des objets Active Directory et participation à l’administration des tenants Microsoft 365.
\end{highlights}

\textbf{Ingénieur Plateforme DevOps — Projet de Fin d’Études (PFE), RFC} \hfill Février 2025 -- Août 2025
\begin{highlights}
\item Conception et développement d’une plateforme DevOps-as-a-Service pour automatiser l’analyse, le déploiement et la supervision des applications.
\item Architecture modulaire basée sur React, Express.js, FastAPI et MongoDB.
\item Intégration de l’API GitHub pour l’analyse des dépôts et workflows.
\item Conception et mise en œuvre de pipelines CI/CD avec Jenkins.
\item Automatisation du provisionnement avec Terraform et Ansible.
\item Conteneurisation avec Docker et gestion des artefacts via Nexus.
\item Mise en place de l’observabilité avec Prometheus et Grafana.
\item Réduction du temps de provisionnement d’environ 70\%.
\end{highlights}

\textbf{Ingénieur Cloud \& Full-Stack — Stage, RFC} \hfill Juillet 2024 -- Août 2024
\begin{highlights}
\item Développement d’applications web cloud-native déployées sur AWS.
\item Mise en place de pipelines CI/CD avec Jenkins, Docker, Terraform et GitHub Actions.
\item Implémentation des contrôles IAM et de la supervision via CloudWatch.
\end{highlights}

\textbf{Développeur Web — Stage, ESPRIT} \hfill Juillet 2023 -- Août 2023
\begin{highlights}
\item Développement d’une plateforme académique Full-Stack basée sur la stack MERN.
\item Implémentation de fonctionnalités avancées de recherche et indexation avec MongoDB.
\item Développement backend avec Express.js et Node.js.
\item Scripts de web scraping Python pour l’ingestion automatisée de données.
\item Conception d’interfaces responsives et amélioration de l’expérience utilisateur.
\end{highlights}

\textbf{Stagiaire Ingénieur Réseau (C-NOC) — Tunisie Telecom} \hfill Août 2021 -- Septembre 2021
\begin{highlights}
\item Travail au sein du centre national d’exploitation réseau B2B.
\item Supervision des liaisons entreprises et gestion des incidents.
\item Assistance au diagnostic des problèmes de routage et de disponibilité.
\end{highlights}

% ================= PROJECTS =================
\section{Projets Académiques \& Techniques}

\textbf{Plateforme de Supervision d’Infrastructure — RFC}
\begin{highlights}
\item Conception et déploiement de tableaux de bord Grafana pour environnements cloud et infrastructure.
\item Mise en place de stratégies d’alerting pour améliorer la réactivité opérationnelle.
\end{highlights}

\textbf{Plateforme CI/CD DevOps — ESPRIT}
\begin{highlights}
\item Chaîne CI/CD complète avec Jenkins, Docker, Docker Compose, SonarQube et Nexus.
\item Déploiement Angular (frontend) et Spring Boot (backend).
\item Supervision via Prometheus et Grafana.
\end{highlights}

\textbf{Laboratoire Cloud Privé OpenStack — ESPRIT}
\begin{highlights}
\item Déploiement d’un cloud privé OpenStack multi-nœuds.
\item Interconnexion via VPN et configuration des réseaux virtuels.
\item Exploitation de machines virtuelles et d’applications Full-Stack.
\end{highlights}

% ================= EDUCATION =================
\section{Formation}
\textbf{Diplôme d’Ingénieur en Génie Logiciel — ESPRIT} \hfill 2020 -- 2025 \\
Ariana, Tunisie

% ================= LANGUAGES =================
\section{Langues}
Arabe (Langue maternelle) \;|\; Anglais (B2) \;|\; Français (B2)

\end{document}
