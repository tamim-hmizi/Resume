\documentclass[10pt, letterpaper]{article}

% ================= PACKAGES =================
\usepackage[
    ignoreheadfoot,
    top=2cm,
    bottom=2cm,
    left=2cm,
    right=2cm
]{geometry}

\usepackage{titlesec}
\usepackage{enumitem}
\usepackage{hyperref}
\usepackage{needspace}
\usepackage{iftex}
\usepackage{charter}

% ================= ATS / ENCODAGE (TOUJOURS) =================
\usepackage[T1]{fontenc}
\usepackage[utf8]{inputenc}
\usepackage{lmodern}
\usepackage[final]{microtype} % améliore l’extraction du texte PDF / réduit les artefacts

% Assurer une lecture machine / ATS :
\input{glyphtounicode}
\pdfgentounicode=1

\pagestyle{empty}
\setcounter{secnumdepth}{0}
\setlength{\parindent}{0pt}
\renewcommand\labelitemi{$\bullet$}

\titleformat{\section}{\bfseries\large}{}{0pt}{}[\titlerule]
\titlespacing{\section}{0pt}{6pt}{6pt}

\newenvironment{highlights}{
\begin{itemize}[leftmargin=12pt, itemsep=2pt]
}{
\end{itemize}
}

\hypersetup{
  colorlinks=true,
  urlcolor=black
}

% ================= DOCUMENT =================
\begin{document}

% ================= EN-TÊTE =================
\begin{center}
{\fontsize{22}{22}\selectfont \textbf{Tamim Hmizi}} \\[4pt]
{\large Consultant en Infrastructure \,|\, Cloud \& DevOps \,|\, SaaS \& Full-Stack} \\[6pt]
Ariana, Tunisie \;|\;
\href{mailto:tamimhmizi@icloud.com}{tamimhmizi@icloud.com} \;|\;
+216 21 611 816 \;|\;
\href{https://linkedin.com/in/tamimhmizi}{linkedin.com/in/tamimhmizi} \;|\;
\href{https://github.com/tamim-hmizi}{github.com/tamim-hmizi} \;|\;
\href{https://tamim-hmizi.github.io/Portfolio/}{Portfolio}
\end{center}

% ================= PROFIL =================
\section{Profil Professionnel}
Ingénieur Cloud \& DevOps et Consultant en Infrastructure avec une expérience dans la livraison d’applications SaaS et full-stack.
Capable de gérer l’ensemble du cycle de vie des systèmes logiciels modernes, depuis la conception et l’automatisation de l’infrastructure jusqu’au CI/CD, au déploiement, au monitoring et aux opérations en production.
Solide expertise dans les environnements hybrides et cloud-native, les pipelines DevOps, l’observabilité, la sécurité et les systèmes IT d’entreprise supportant des applications scalables.

% ================= COMPÉTENCES =================
\section{Compétences Techniques Clés}
\textbf{Cloud \& Plateformes :} Azure, AWS, GCP, OpenStack, DigitalOcean, Cloud Hybride, IaaS, PaaS \\
\textbf{Conteneurs \& Orchestration :} Docker, Docker Compose, Kubernetes, Helm \\
\textbf{DevOps \& Automatisation :} Jenkins, GitHub Actions, GitLab CI, Terraform, Ansible, Pipelines CI/CD, Infrastructure as Code (IaC), Nexus \\
\textbf{Monitoring \& Observabilité :} Grafana, Prometheus, ELK Stack, CloudWatch, Alerting, Métriques, Tableaux de bord \\
\textbf{Sécurité \& Qualité :} Pare-feux Fortinet, VPN, Segmentation réseau, IAM, SonarQube, Trivy, OWASP \\
\textbf{Sauvegarde \& PRA :} Commvault (jobs de sauvegarde, tests de restauration, analyse des échecs, reporting) \\
\textbf{Systèmes \& Identité :} Linux, Windows Server, Active Directory, Microsoft 365 \\
\textbf{Langages de Programmation :} Python, Bash, Java, JavaScript, TypeScript, C, C++, C\#, PHP, Assembleur, SQL \\
\textbf{Frameworks Backend :} FastAPI, Django, Flask, Express.js, Spring Boot, Node.js, APIs REST \\
\textbf{Frameworks Frontend :} Angular, React, Next.js, Vue.js, HTML, CSS \\
\textbf{Bases de Données :} MongoDB, MySQL, PostgreSQL, DynamoDB, Redis \\
\textbf{Contrôle de Version :} Git, GitHub, GitLab, Bitbucket, Gitea

% ================= EXPÉRIENCE =================
\section{Expérience Professionnelle}

\textbf{Consultant en Infrastructure — RFC} \hfill Oct 2025 -- Présent \\
Ariana, Tunisie
\begin{highlights}
\item Exploitation et support d’une \textbf{infrastructure cloud hybride de production} basée sur \textbf{Azure Stack Hub intégré à Azure public} pour la fourniture de services SaaS et d’entreprise.
\item Administration et dépannage de l’\textbf{infrastructure réseau et sécurité d’entreprise} à l’aide de \textbf{pare-feux Fortinet} et d’\textbf{équipements réseau Cisco}.
\item Mise en œuvre de la protection des endpoints et serveurs avec \textbf{l’antivirus Trend Micro} sur les environnements d’infrastructure.
\item Exploitation et maintenance de l’\textbf{infrastructure de sauvegarde Commvault}, incluant l’exécution des sauvegardes, la validation des restaurations, l’analyse des échecs et le reporting.
\item Support infrastructure via la gestion des incidents, le troubleshooting et le traitement des demandes de service.
\item Animation d’ateliers techniques et de sessions de formation clients axées sur l’utilisation de l’infrastructure, la sécurité et la stratégie de sauvegarde.
\item Gestion des services \textbf{Active Directory} et participation à l’administration des tenants \textbf{Microsoft 365}.
\end{highlights}

\textbf{Ingénieur DevOps Plateforme — Projet de Fin d’Études (PFE), RFC} \hfill Fév 2025 -- Août 2025
\begin{highlights}
\item Conception et implémentation d’une \textbf{plateforme SaaS DevOps-as-a-Service} pour automatiser l’analyse, le déploiement et le monitoring des applications.
\item Développement d’une \textbf{application cœur monolithique} utilisant \textbf{React}, \textbf{Express.js}, \textbf{FastAPI} et \textbf{MongoDB}, avec une séparation claire entre frontend, API gateway et services backend.
\item Intégration d’un \textbf{moteur d’analyse piloté par l’IA (basé LLM)} pour inspecter les dépôts Git, la structure des applications, les dépendances et les exigences d’exécution.
\item Implémentation d’une \textbf{logique de déploiement multi-architectures}, permettant une prise de décision automatisée entre :
\begin{itemize}[leftmargin=16pt, itemsep=2pt]
  \item \textbf{Déploiements sur machines virtuelles} pour les applications monolithiques et one-tier.
  \item \textbf{Déploiements Kubernetes (AKS)} pour les architectures orientées microservices.
\end{itemize}
\item Développement de deux \textbf{applications e-commerce utilisant la stack MERN} comme cas réels de déploiement :
\begin{itemize}[leftmargin=16pt, itemsep=2pt]
  \item Une \textbf{application e-commerce monolithique one-tier} déployée sur \textbf{Machine Virtuelle}, sélectionnée et recommandée par le moteur IA selon la structure du dépôt et l’analyse runtime.
  \item Un \textbf{backend e-commerce orienté microservices} déployé sur \textbf{Kubernetes (AKS)}, recommandé par le moteur IA en raison de la décomposition des services et des exigences de scalabilité.
\end{itemize}
\item Automatisation des \textbf{déploiements sur VM} avec \textbf{Terraform} pour le provisioning, \textbf{Ansible} pour la configuration et \textbf{Docker / Docker Compose} pour la livraison applicative.
\item Implémentation des \textbf{déploiements microservices AKS} avec \textbf{AKS Engine}, \textbf{kubectl} et des \textbf{manifests YAML Kubernetes} par microservice.
\item Conception et mise en place de \textbf{pipelines CI/CD avec Jenkins}, couvrant les étapes de build, test, qualité, sécurité, gestion des artefacts et déploiement.
\item Intégration de \textbf{SonarQube} pour l’analyse de la qualité du code et de \textbf{Nexus} pour la gestion des artefacts.
\item Mise en œuvre complète de l’\textbf{observabilité et du monitoring} avec \textbf{Prometheus} et \textbf{Grafana}.
\item Réduction du temps de provisioning et de déploiement de l’infrastructure d’environ \textbf{70\%} grâce à l’automatisation.
\end{highlights}

\textbf{Ingénieur Cloud \& Full Stack — Stage, RFC} \hfill Juil 2024 -- Août 2024
\begin{highlights}
\item Développement d’une \textbf{application web cloud-native} pour la \textbf{gestion des stagiaires}, couvrant l’onboarding, le suivi et les workflows administratifs.
\item Déploiement de l’application sur \textbf{AWS EKS (Kubernetes)} avec une architecture entièrement conteneurisée.
\item Conception et implémentation de \textbf{pipelines CI/CD de bout en bout} avec \textbf{Jenkins}, \textbf{Docker}, \textbf{Terraform} et \textbf{GitHub Actions}.
\item Mise en œuvre des \textbf{contrôles de sécurité IAM} et du \textbf{monitoring applicatif et infrastructure} avec \textbf{Prometheus} et \textbf{Grafana}.
\end{highlights}

\textbf{Développeur Web — Stage, ESPRIT} \hfill Juil 2023 -- Août 2023
\begin{highlights}
\item Conception et développement d’une \textbf{plateforme web de scraping de sites éducatifs officiels} pour collecter et agréger des cours en ligne.
\item Implémentation de \textbf{modules de web scraping en Python} pour extraire les métadonnées des cours (titre, fournisseur, durée, niveau).
\item Normalisation et stockage des données collectées dans une base structurée pour une recherche et un filtrage efficaces.
\item Développement de services backend pour classer et recommander les \textbf{meilleurs cours selon les critères de recherche des utilisateurs}.
\item Création d’une interface web responsive permettant la recherche, la comparaison et la découverte de cours pertinents issus de sources fiables.
\item Garantie de la fiabilité des données en ciblant uniquement des \textbf{fournisseurs de cours officiels et vérifiés}.
\end{highlights}

\textbf{Stagiaire Ingénieur Réseau (C-NOC) — Tunisie Telecom} \hfill Août 2021 -- Sep 2021
\begin{highlights}
\item Travail au sein du centre national B2B de supervision réseau.
\item Supervision des liaisons réseau d’entreprise et gestion des incidents de connectivité.
\item Assistance au dépannage des problèmes de routage et de disponibilité des services.
\end{highlights}

% ================= PROJETS =================
\section{Projets Académiques \& Techniques}

\textbf{Plateforme de Monitoring Entreprise — RFC}
\begin{highlights}
\item Conception et implémentation de \textbf{tableaux de bord Grafana personnalisés} pour la supervision des environnements cloud et infrastructure.
\item Développement d’un \textbf{exporter Python personnalisé} communiquant avec les \textbf{APIs Azure Stack Hub} via une \textbf{application Azure AD} pour une authentification sécurisée.
\item Collecte des métriques infrastructure et plateforme via l’exporter et exposition au format \textbf{compatible Prometheus}.
\item Intégration de \textbf{Prometheus} pour le scraping, le stockage des métriques et l’évaluation des règles d’alerting.
\item Visualisation des métriques et alertes dans des tableaux de bord \textbf{Grafana} afin d’améliorer la visibilité opérationnelle et les temps de réponse.
\item Livraison d’une solution de monitoring entièrement \textbf{pilotée par le code}, avec tous les composants \textbf{conteneurisés via Docker}.
\end{highlights}

\textbf{Plateforme CI/CD DevOps — ESPRIT}
\begin{highlights}
\item Mise en place d’un pipeline CI/CD de bout en bout avec Jenkins, Docker, Docker Compose, SonarQube et Nexus.
\item Déploiement d’applications frontend Angular et backend Spring Boot.
\item Implémentation du monitoring avec Prometheus et Grafana.
\end{highlights}

\textbf{Laboratoire Cloud Privé OpenStack — ESPRIT}
\begin{highlights}
\item Conception et déploiement d’un cloud privé OpenStack multi-nœuds dans le cadre de projets académiques.
\item Interconnexion de nœuds physiques via VPN et mise en place de réseaux virtuels, sous-réseaux et routage.
\item Déploiement et exploitation de machines virtuelles et d’applications web full-stack dans l’environnement OpenStack.
\end{highlights}

% ================= FORMATION =================
\section{Formation}
\textbf{Diplôme d’Ingénieur en Génie Logiciel — ESPRIT} \hfill 2020 -- 2025 \\
Ariana, Tunisie

% ================= LANGUES =================
\section{Langues}
Arabe (Langue maternelle) \;|\; Anglais (B2) \;|\; Français (B2)

\end{document}
